\documentclass[12pt]{article}

\usepackage[utf8]{inputenc}
\usepackage[brazil]{babel}
\usepackage[T1]{fontenc}
\usepackage{enumitem}
\usepackage{graphicx}
\usepackage{bookmark}
\usepackage{textcomp}
\usepackage{tikz}
\usepackage{subfigure}


\usepackage{hyperref}
\hypersetup{
	pdftitle    = {Exercícios de Estrutura de Dados II},
	pdfsubject  = {Preparação para a primeira prova},
	pdfauthor   = {Paulo Roberto Urio},
	pdfcreator  = {Paulo Roberto Urio},
	pdfproducer = {Paulo Roberto Urio},
	pdfkeywords = {árvore} {exercícios} {prova}
}

\setcounter{secnumdepth}{0}
\newcommand{\itembf}[1]{ \item{ \textbf{#1} }  \\ }

\begin{document}


\begin{center}
Universidade Estadual do Centro Oeste do Paraná \\
Departamento de Ciência da Computação \\
Algoritmos e Estrutura de Dados II \\
Professora Tony Alexander Hild \\
Aluno/RA: Paulo Roberto Urio - 570091403
\end{center}

\section{Árvore AVL}

\begin{enumerate}

\itembf{O que é o fator de balanceamento de um nó?}
	O fator de balanceamento de um nó é dado pelo seu peso em 
	relação a sua sub-árvore.

\itembf{Mostre com exemplos que cada nó de uma árvore AVL tem 
	balanceamento -1, 0 ou 1.}
	Em pré-ordem:
	\begin{figure}[ht]
		\subfigure[111] {
			\includegraphics[scale=0.4]{arvores/ex1.svg}
		}

		\subfigure[222] {
			\includegraphics[scale=0.4]{arvores/ex2.svg}
		}

		\subfigure[333] {
			\includegraphics[scale=0.4]{arvores/ex3.svg}
		}
	\end{figure}


\end{enumerate}

\end{document}

